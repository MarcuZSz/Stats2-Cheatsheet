\documentclass[10pt,a4paper,landscape]{article}
\usepackage[utf8]{inputenc}
\usepackage[german]{babel}
\usepackage[T1]{fontenc}
\usepackage{amsmath}
\usepackage{amsfonts}
\usepackage{amssymb}
\usepackage{graphicx}
\usepackage{lmodern}
\usepackage{fourier}
\usepackage{array}
\usepackage{array}
\usepackage{multirow}
\usepackage{xcolor}
\usepackage{tcolorbox}
\usepackage{titlesec}
\usepackage{pdfpages}
\usepackage{pgffor}


\usepackage{multicol}
\setlength{\columnsep}{0.5cm}

\usepackage{extsizes}
\newcommand{\setdocumentfont}{\fontsize{6}{8}\selectfont} 


% Schriftgrößenänderung für Section-Titel
\titleformat{\section}[block]{\normalfont\fontsize{8}{10}\bfseries\textcolor{red}}{\thesection}{1em}{}

\titleformat{\subsection}[block]{\normalfont\fontsize{7}{9}\bfseries}{\thesubsection}{1em}{}

\titleformat{\subsubsection}[block]{\normalfont\fontsize{6}{8}\bfseries}{\thesubsubsection}{1em}{}

\usepackage[left=1cm,right=1.2cm,top=1cm,bottom=1cm]{geometry}

\tcbset{
  sectionbox/.style={
    colframe=blue!50!black,
    colback=blue!5,
    coltitle=blue!20!black,
    fonttitle=\bfseries,
    boxrule=0.5mm,
    rounded corners,
    breakable,
    width=\linewidth,
  }
}

\pagestyle{empty}
\title{shitSheat}
\begin{document}
\setdocumentfont
\begin{multicols}{4}
\noindent

\section{Grundbegriffe}
Grundgesamtheit, Population\\
	(gesamte Gruppe, Bsp. Schüler einer Schule)

\begin{enumerate}

\item Umfang
\begin{itemize}
\item Stichprobe/Teilerhebung\\
	(nur kleiner Teil der gesamten Gruppe untersucht)
\end{itemize}

\end{enumerate}

%=========================================================================================================================================================================================================================

\section{Grundlagen: Wahrscheinlichkeit}

\subsubsection*{Wichtiges zur Wahrscheinlichkeitsrechnung}
\begin{itemize}
\item $P(\cup _{i=1} ^{n}) = \sum_{i=1}^n P(A_i)$ für paarweise disjunkte Ereignisse $A_1, A_2, ~..., A_n \subset \Omega$
\item $A \subseteq B \Rightarrow P(A) \leqslant P(B)$
\item \(P (\overline{A}) = 1 - P(A)\)
\item $P(A|\overline{B})= 1 - P(\overline{A} |\overline{B})$
\item $P(\overline{B}\cap A)=P(A|\overline{B})\cdot P(\overline{B})$
\item $P(A\cup B) = P(A) + P(B) -P(A\cap B)$ für beliebige A, B $\subset \Omega$
\item $P(A \cup B) = P(A) + P(B)$ für disjunkte Ereignisse $A,B \subseteq \Omega$ (Additivität)
\end{itemize}

\subsection{Bedingte Wahrscheinlichkeiten}
\[
P(A|B) := \frac{P(A\cap B)}{P(B)}
\]

\subsubsection*{Satz der totalen Wahrscheinlichkeit}
\[
P(A)=\sum_{i=1}^n P(A|B_i) \cdot P(B_i)
\]

\subsubsection*{Wichtiger Spezialfall: Multiplikationssatz}
\[
P(A)=P(A|B)P(B)+P(A|\overline{B})P(\overline{B})
\]

\subsection{Stochastische Unabhängigkeit}
\[
P(A|B) = P(A) ~bzw.~ P(B|A)=P(B)
\]
\[
P(A \cap B)=P(A) \cdot P(B)
\]

\subsection{Bedingte Unabhängigkeit}
\[
P(A \cap B|C)=P(A|C)\cdot P(B|C)
\]

\subsection{Satz von Bayes}
\[ \resizebox{0.15\textwidth}{!}{$
P(A|B)=\frac{P(A\cap B)}{P(B)} \Rightarrow P(A\cap B)=P(A|B)P(B)
$}
\]

\[
\Rightarrow P(A|B)P(B)=P(B|A)P(A)
\]

\subsection{Satz von Bayes II}
\[ \resizebox{0.15\textwidth}{!}{$
P(B|A)=\frac{P(A|B)P(B)}{P(A)}=\frac{P(A|B)P(B)}{P(A|B)P(B)+P(A|\overline{B})P(\overline{B})}
$}
\]
\[ 
P(B_i|A)=\frac{P(A|B_i)P(B_i)}{\sum_{j=1}^n P(A|B_j)P(B_j)}
\]

%=========================================================================================================================================================================================================================

\section{Zusammenhangsmaße diskreter Merkmale}
X = Zeilen \\
Y = Spalten
\subsection{Kontigenztafeln}
\begin{itemize}
\item Absolute Häufigkeit: n = n
\item Relative Häufigkeiten: n = 1
\end{itemize}

\subsubsection*{Bedingte Häufigkeiten}

\[
f_{Y|X}(b_1|a_1)=\frac{h_{i1}}{h_{i\cdot}},~...,~ f_{Y|X}(b_m|a_i)=\frac{h_{im}}{h{i\cdot}}
\]

\[
f_{X|Y}(a_1|b_1)=\frac{h_{1j}}{h_{\cdot j}},~...,~ f_{X|Y}(a_k|b_j)=\frac{h_{kj}}{h{\cdot j}}
\]

\subsection{Empirische Unabhängigkeit}
\[ \resizebox{0.2\textwidth}{!}{$
f_{Y|X}(b_j|a_1)=f_Y(b_j|a_2)=...=f_{Y|X}(b_j|a_k)\forall j=1,...,m
$}
\]
oder halt:
\[
\forall i,j: ~ h_{ij}=\tilde{h}_{ij}
\]

\subsection{$\chi^2$- und Kontigenzkoeffizient}

\subsubsection*{Berechnung von $\tilde{h}$}
ist erwartete Häufigkeit in einer Kontingenztafel
\[
\text{mit}~~ \tilde{h}_{ij}=\frac{h_{i\cdot}h_{\cdot j}}{n}
\]

\subsubsection*{$\chi^2$-Koeffizient}

{\tiny
\begin{align*}
\chi^2 := \sum_{i=1}^k \sum_{j=1}^m \frac{(h_{ij}-\tilde{h}_{ij})^2}{\tilde{h}_{ij}}&=\sum_{i=1}^k \sum_{j=1}^m \frac{(h_{ij}-\frac{h_{i\cdot}h_{\cdot j}}{n})^2}{\frac{h_{i\cdot}h_{\cdot j}}{n}} \\
&=  n\sum_{i} \sum_{j}\frac{(f_{ij}-f_{i\cdot}f_{\cdot j})^2}{f_{i\cdot}f_{\cdot j}} \\
\text{mit}~~ \tilde{h}_{ij}=\frac{h_{i\cdot}h_{\cdot j}}{n}
\end{align*}
}

\begin{itemize}
\item $\chi^2 \in [0, n(min(k,m)-1)]$
\item $\chi^2=0 \Leftrightarrow$ X und Y empirisch unabhängig
\item $\chi^2$ groß, starker Zusammenhang
\item $\chi^2$ klein, schwacher Zusammenhang
\end{itemize}

\subsubsection*{Spezialfall für 2x2-Tafel}
\[ \resizebox{0.09\textwidth}{!}{$
\begin{array}{|cc|c|}
\hline
a & b & a+b \\
c & d & c+d \\
\hline
a+c & b+d & \\
\hline
\end{array}
$}
\]

\[
\chi^2 = \frac{n(ad-bc)^2}{(a+b)(a+c)(b+d)(c+d)}
\]

\subsubsection*{Kontigenzkoeffizient}
\[ \resizebox{0.15\textwidth}{!}{$
K := \sqrt{\frac{\chi^2}{n+\chi^2}}, ~K \in \left[0, \sqrt{\frac{M-1}{M}}\right],~ M=min\{k, m\}
$}
\]

\subsubsection*{Korrigierter Koeffizient}
\[
K^* := \frac{K}{\sqrt{(M-1)/M}}, ~ K^* \in [0, 1]
\]

%=========================================================================================================================================================================================================================

\section{ZVs, Verteilungen und Häufigkeiten}

\subsection{Zufallsvariable}
\[
X:\Omega \rightarrow T_X , ~ T_X\subseteq \mathbb{R}
\]

\begin{itemize}
\item ordnet jedem $\omega \in \Omega$ genau ein $x \in T_X$ zu: $X(\omega)=x$
\item Mehrere Elementarereignisse können selben Zahlenwert zugeordnet sein
\item Ausprägungen/Relisierungen: $x=X(\omega)$ von X
\end{itemize}

\subsection{Diskreter Erwartungswert}
\begin{align*}
E(X) :&=\sum_{x\in T_X} x\cdot P(X=x) = \sum_{\omega \in \Omega} P(\{\omega\})X(\omega) \\
&= \sum_{x\in T_X} x\cdot f_X(x)
\end{align*}

\subsection{Eigenschaften Erwartungswert}
\begin{itemize}
\item X=a mit Wahrscheinlichkeit 1 (determinische ZV)
\[
E(X)=a
\]
\item Linearität
\begin{align*}
E(a\cdot X + b \cdot Y) &= a\cdot E(X) + b \cdot	E(Y) \\
E(aX+b) &= a\cdot E(X) + b
\end{align*}
\item symmetrisch um Punkt c: $f(c-x)=f(c+x) \forall x \in T_X$, dann $E(X)=c$
\item für bel. $a_1,...,a_n \in \mathbb{R}$ und bel. ZV $X_1,...,X_n$
\[
E\left(\sum_{i=1}^na_iX_i\right)=\sum_{i=1}^na_i \cdot E(X_i)
\]
\end{itemize}

\subsection{Transformationsregel Erwartungswert}
Für reelle Funktion Y=g(X):
\begin{align*}
E(Y)=E[g(X)]=\begin{cases}
\sum_{x\in T}g(x)f(x) &\text{X diskret} \\
\int_{-\infty}^{\infty}g(x)f(x)dx &\text{X stetig}
\end{cases}
\end{align*}

\subsection{Varianz ZV}
anstatt $Var(X)$ geht auch $\sigma^2$
\[
Var(X) = \sum_{i=1}^n (x_i - E(X))^2P(X=x_i)
\]

\subsection*{Eigenschaften Varianz}
\begin{itemize}
\item Einfachere Berechnung mit Varianz:
\[
Var(X)=E(X^2)-(E(X))^2
\]
\item $Var(aX+b)=a^2Var(X) \forall a,b \in \mathbb{R}$
\item unabhängige ZV X, Y: $Var(X+Y)=Var(X)+Var(Y)$
\end{itemize}

\subsection{Standardabweichung}
\[
\sigma = + \sqrt{Var(X)}
\]

%=========================================================================================================================================================================================================================

\section{Wichtige parametrische Verteilungen}

\subsubsection{Binomialverteilung}
Modelliert Anzahl Erfolge mit fester Anzahl Versuchen\\
$E(X)=np$ und $Var(X)=np(1-p)$
\[
P(X=x)=f(x)=\binom{n}{k} \cdot p^k(1-p)^{n-k}, ~ k \in T 
\]
\[
X \sim \mathcal{B}(n, p), ~ n \in \mathbb{N}, ~p \in [0,1]
\]

\subsubsection{Negative Binomialverteilung}
Verwendung: Wartezeiten/Misserfolge; "alle Fehlerfolge vor Erfolgen" \\
$T_X=\{n, n+1, n+2, ...\}$ mit $n \in \mathbb{N}^+, \pi \in (0,1)$
\[\resizebox{0.2\textwidth}{!}{$
f(x)=\binom{x-1}{n-1}\pi ^n(1-\pi)^{x-n}|(x \geq n), \pi \in (0,1), T_X=\{n, n+1, ...\} ~ n \in \mathbb{N}^+
$}
\]
\[
X \sim \mathcal{NB}(n, \pi)
\]

\subsubsection{Geometrische Verteilung}
Modelliert Anzahl Versuche bis A zum erster Erfolg eintritt\\
$E(X)=\frac{1}{p}$ und $Var(X)=\frac{1-p}{p^2}$
\[
f(x)= \underbrace{(1 - p)^{x-1}}_{\text{(x-1)-mal } \overline{A}}
 \cdot \underbrace{p}_{\text{1-Mal A}}|(x \in T), p \in (0,1)
\]
\[
X \sim \mathcal{G}(p)
\]

\subsubsection{Normalverteilung}
modelliert kontinuierliche Werte um Mittelwert mit sym. Verteilung (Körpergröße, Gewicht) \\
Träger $T=\mathbb{R}$ mit Parametern $\mu \in \mathbb{R}, \sigma^2 \in \mathbb{R}_+$
\[
f(x)=\frac{1}{\sqrt{2 \pi \sigma^2}}exp\left(-\frac{1}{2}\frac{(x-\mu)^2}{\sigma^2}\right), x\in \mathbb{R}
\]
$\mu = 0$ und $\sigma^2 = 1$ ist standardnormalverteilt.
\[
X \sim \mathcal{N}(\mu , \sigma^2)
\]

\subsubsection{Poisson-Verteilung}
Modelliert Anzahl Ereignisse mit festen Zeit/Raumbereich $\rightarrow$ konstante Rate, unabhängig (Anzahl Anrufe Callcenter)
\[
f(x)=exp(-\lambda)\frac{\lambda^x}{x!}|(x \in T), t= \mathbb{N^+}
\]
\[
X \sim \mathcal{P}(\lambda)
\]

Approximation:
\[
\mathcal{B}(n, \pi) \approx \mathcal{P}(\lambda =n\pi)
\]

%=========================================================================================================================================================================================================================

\section{Schätzung und Grenzwertsätze}

\subsection{Standadisierte ZV/Z-Score}
\begin{align*}
\tilde{X} &= \frac{X- \mu_X}{\sigma_X} \\
E(\tilde{X}) &= (E(X)-\mu_X)=0 \\
Var(\tilde{X})&=\frac{1}{\sigma^2_X}Var(X) = 1
\end{align*}

\subsection{Standadisierung summierter ZVn}
\begin{align*}
E(Y_n) = n \cdot \mu_X, ~ Var(Y_n)= n \cdot \sigma_X^2 \\
Z_n = \frac{Y_n - n\mu_X}{\sqrt{n}\sigma_X}=\frac{1}{\sqrt{n}}\sum_{i=1}^n \frac{X_i-\mu_X}{\sigma_X}
\end{align*}

\subsection{Zentraler Grenzwertsatz}
\[
\mu_{\hat{p}}=p, ~SE_{\hat{p}}=\sqrt{\frac{p(1-p)}{n}}
\]
\[
Z_n \sim \mathcal{N}(\mu = 0, \sigma^2 = 1) ~ \text{a :$\cong$ asymptotisch}
\]

Nett to know:
\[
\mu_x = \mu, ~ \sigma_x = \frac{\sigma}{\sqrt{n}}
\]

\subsection{Punktschätzung}
\begin{enumerate}
\item Z-Score/Standadisieren
\item Tabellenwerk
\item Tabellenwert-(1-Tabellenwert)
\end{enumerate}

\subsection{Intervallschätzung}
\[
\hat{p} \pm 1.96 \cdot SE_{\hat{p}}
\]

\end{multicols}
\end{document} 