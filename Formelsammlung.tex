\documentclass[10pt,a4paper,landscape]{article}
\usepackage[utf8]{inputenc}
\usepackage[german]{babel}
\usepackage[T1]{fontenc}
\usepackage{amsmath}
\usepackage{amsfonts}
\usepackage{amssymb}
\usepackage{graphicx}
\usepackage{lmodern}
\usepackage{fourier}
\usepackage{array}
\usepackage{array}
\usepackage{multirow}
\usepackage{xcolor}
\usepackage{tcolorbox}
\usepackage{titlesec}
\usepackage{pdfpages}
\usepackage{pgffor}

\usepackage{multicol}
\setlength{\columnsep}{0.5cm}

\usepackage{extsizes}
\newcommand{\setdocumentfont}{\fontsize{6}{8}\selectfont} 


% Schriftgrößenänderung für Section-Titel
\titleformat{\section}[block]{\normalfont\fontsize{8}{10}\bfseries\textcolor{red}}{\thesection}{1em}{}

\titleformat{\subsection}[block]{\normalfont\fontsize{7}{9}\bfseries}{\thesubsection}{1em}{}

\titleformat{\subsubsection}[block]{\normalfont\fontsize{6}{8}\bfseries}{\thesubsubsection}{1em}{}

\usepackage[left=1cm,right=1.2cm,top=1cm,bottom=1cm]{geometry}

\tcbset{
  sectionbox/.style={
    colframe=blue!50!black,
    colback=blue!5,
    coltitle=blue!20!black,
    fonttitle=\bfseries,
    boxrule=0.5mm,
    rounded corners,
    breakable,
    width=\linewidth,
  }
}

\pagestyle{empty}
\title{shitSheat}
\begin{document}
\setdocumentfont
\begin{multicols}{4}
\noindent

\section{Grundbegriffe}

\begin{itemize}
\item Grundgesamtheit, Population\\
	(gesamte Gruppe, Bsp. Schüler einer Schule)
\item Stichprobe/Teilerhebung\\
	(nur kleiner Teil der gesamten Gruppe untersucht)
\end{itemize}



%=========================================================================================================================================================================================================================

\section{Grundlagen: Wahrscheinlichkeit}

\subsubsection*{Wichtiges zur Wahrscheinlichkeitsrechnung}
\begin{itemize}
\item $P(\cup _{i=1} ^{n}) = \sum_{i=1}^n P(A_i)$ für paarweise disjunkte Ereignisse $A_1, A_2, ~..., A_n \subset \Omega$
\item $A \subseteq B \Rightarrow P(A) \leqslant P(B)$
\item \(P (\overline{A}) = 1 - P(A)\)
\item $P(A|\overline{B})= 1 - P(\overline{A} |\overline{B})$
\item $P(\overline{B}\cap A)=P(A|\overline{B})\cdot P(\overline{B})$
\item $P(A\cup B) = P(A) + P(B) -P(A\cap B)$ für beliebige A, B $\subset \Omega$
\item $P(A \cup B) = P(A) + P(B)$ für disjunkte Ereignisse $A,B \subseteq \Omega$ (Additivität)
\item Sensitivität: $TPR=P(B|A)$
\item Spezifität: $TNR=P(\bar{B}|A)= 1- FPR$
\item $FPR = P(B|\bar{A})$
\end{itemize}

\subsection{Bedingte Wahrscheinlichkeiten}
\[
P(A|B) := \frac{P(A\cap B)}{P(B)}
\]

\subsubsection*{Satz der totalen Wahrscheinlichkeit}
\[
P(A)=\sum_{i=1}^n P(A|B_i) \cdot P(B_i)
\]

\subsubsection*{Wichtiger Spezialfall: Multiplikationssatz}
\[
P(A)=P(A|B)P(B)+P(A|\overline{B})P(\overline{B})
\]

\subsection{Stochastische Unabhängigkeit}
\[
P(A|B) = P(A) ~bzw.~ P(B|A)=P(B)
\]
\[
P(A \cap B)=P(A) \cdot P(B)
\]

\subsection{Bedingte Unabhängigkeit}
\[
P(A \cap B|C)=P(A|C)\cdot P(B|C)
\]

\subsection{Satz von Bayes}
\[ \resizebox{0.15\textwidth}{!}{$
P(A|B)=\frac{P(A\cap B)}{P(B)} \Rightarrow P(A\cap B)=P(A|B)P(B)
$}
\]

\[
\Rightarrow P(A|B)P(B)=P(B|A)P(A)
\]

\subsection{Satz von Bayes II}
\[ \resizebox{0.15\textwidth}{!}{$
P(B|A)=\frac{P(A|B)P(B)}{P(A)}=\frac{P(A|B)P(B)}{P(A|B)P(B)+P(A|\overline{B})P(\overline{B})}
$}
\]
\[ 
P(B_i|A)=\frac{P(A|B_i)P(B_i)}{\sum_{j=1}^n P(A|B_j)P(B_j)}
\]

%=========================================================================================================================================================================================================================

\section{Zusammenhangsmaße diskreter Merkmale}

\subsection{Kontigenztafeln}
\begin{itemize}
\item Absolute Häufigkeit: n = n
\item Relative Häufigkeiten: n = 1
\end{itemize}

\subsubsection*{Bedingte Häufigkeiten}

\[
f_{Y|X}(b_1|a_1)=\frac{h_{i1}}{h_{i\cdot}},~...,~ f_{Y|X}(b_m|a_i)=\frac{h_{im}}{h{i\cdot}}
\]

\[
f_{X|Y}(a_1|b_1)=\frac{h_{1j}}{h_{\cdot j}},~...,~ f_{X|Y}(a_k|b_j)=\frac{h_{kj}}{h{\cdot j}}
\]

\subsection{Empirische Unabhängigkeit}
\[ \resizebox{0.2\textwidth}{!}{$
f_{Y|X}(b_j|a_1)=f_Y(b_j|a_2)=...=f_{Y|X}(b_j|a_k)\forall j=1,...,m
$}
\]
oder halt:
\[
\forall i,j: ~ h_{ij}=\tilde{h}_{ij}
\]

\[
\tilde{h}_{ij}=\frac{h_{i \cdot} h_{\cdot j}}{n}
\]

%=========================================================================================================================================================================================================================

\section{ZVs, Verteilungen und Häufigkeiten}

\subsection{Erwartungswert}
\begin{align*}
E(X) :&=\sum_{x\in T_X} x\cdot P(X=x)
\end{align*}

\begin{itemize}
\item X=a mit Wahrscheinlichkeit 1 (determinische ZV)
\[
E(X)=a
\]
\item Linearität
\begin{align*}
E(a\cdot X + b \cdot Y) &= a\cdot E(X) + b \cdot	E(Y) \\
E(aX+b) &= a\cdot E(X) + b
\end{align*}
\item symmetrisch um Punkt c: $f(c-x)=f(c+x) \forall x \in T_X$, dann $E(X)=c$
\item $E\left(\sum_{i=1}^na_iX_i\right)=\sum_{i=1}^na_i \cdot E(X_i)$
\end{itemize}

\subsection{Varianz ZV}
anstatt $Var(X)$ geht auch $\sigma^2$
\[
Var(X) = \sum_{i=1}^n (x_i - E(X))^2P(X=x_i)
\]

\begin{itemize}
\item Einfachere Berechnung mit Varianz:
\[
Var(X)=E(X^2)-(E(X))^2
\]
\item $Var(aX+b)=a^2Var(X) \forall a,b \in \mathbb{R}$
\item unabhängige ZV X, Y: $Var(X+Y)=Var(X)+Var(Y)$
\end{itemize}

\subsection{Standardabweichung}
\[
\sigma = + \sqrt{Var(X)}
\]

%=========================================================================================================================================================================================================================

\section{Wichtige parametrische Verteilungen}

\subsection{Bernoulli-Verteilung}
Modelliert Experiment mit 2 mögl. Ausgängen

\begin{align*}
& P(X=1)=f(1) = \pi
& P(X=0)=f(0) = 1 - \pi
\end{align*}

\subsection{Binomialverteilung}
Modelliert Anzahl Erfolge mit fester Anzahl Versuchen\\
$E(X)=n\pi$ und $Var(X)=n\pi(1-\pi)$
\[
P(X=x)=f(x)=\binom{n}{k} \cdot \pi^x(1-\pi)^{n-x}
\]


\subsection{Negative Binomialverteilung}
Wartezeiten/Misserfolge; alle Fehlerfolge vor Erfolgen \\
$E(X)=\frac{x}{\pi}$ und $Var(X)=\frac{x(1-\pi)}{\pi^2}$
\[\resizebox{0.1\textwidth}{!}{$
f(x)=\binom{n-1}{k-1}\pi ^x(1-\pi)^{n-x}
$}
\]


\subsection{Geometrische Verteilung}
Modelliert Anzahl Versuche bis A zum erster Erfolg eintritt\\
$E(X)=\frac{1}{\pi}$ und $Var(X)=\frac{1-\pi}{\pi^2}$
\[
f(x)= \underbrace{(1 - \pi)^{x-1}}_{\text{(x-1)-mal } \overline{A}}
 \cdot \underbrace{\pi}_{\text{1-Mal A}}
\]

\subsection{Normalverteilung}
modelliert kontinuierliche Werte um Mittelwert mit sym. Verteilung (Körpergröße, Gewicht) \\

\[
f(x)=\frac{1}{\sqrt{2 \pi \sigma^2}}exp\left(-\frac{1}{2}\frac{(x-\mu)^2}{\sigma^2}\right)
\]
$\mu = 0$ und $\sigma^2 = 1$ ist standardnormalverteilt.


\subsection{Poisson-Verteilung}
Modelliert Anzahl Ereignisse mit festen Zeit/Raumbereich $\rightarrow$ konstante Rate, unabhängig (Anzahl Anrufe Callcenter) \\
$E(X)=Var(X)=\lambda$
\[
f(x)=exp(-\lambda)\frac{\lambda^x}{x!}
\]

\subsection{Symmetrie und Schiefe}
\begin{itemize}
\item symmetrisch: \\
rechts und links annähernd gleich

\item linkssteil(rechtsschief): \\
Verteilung nach links steiler 

\item rechtssteil(linksschief): \\
Verteilung nach rechts steiler
\end{itemize}
\begin{center}
\includegraphics[scale=0.2]{Bilder/SymmetrieSchiefe.png}
\end{center}

%=========================================================================================================================================================================================================================

\section{Schätzung und Grenzwertsätze}

\subsection{Standadisierte ZV/Z-Score}
Prüfe Normalverteilungsapprox. $\geq$ 10
\begin{align*}
Z &= \frac{x - \mu}{\sigma} 
\end{align*}


\subsection{Zentraler Grenzwertsatz}
\[
\mu_{\hat{p}}=p, ~\hat{p}=\frac{x}{n}, ~SE_{\hat{p}}=\sqrt{\frac{p(1-p)}{n}}
\]

Nett to know:
\[
\mu_x = \mu, ~ \sigma_x = \frac{\sigma}{\sqrt{n}}
\]

\subsection{Punktschätzung}
\begin{enumerate}
\item Z-Score/Standadisieren
\item Tabellenwerk
\item Tabellenwert-(1-Tabellenwert)
\end{enumerate}

\subsection{Intervallschätzung}
\[
\hat{p} \pm 1.96 \cdot SE_{\hat{p}}
\]

\subsection{Hypothesentest}
\begin{align*}
& H_0 : p = x, H_A: p \neq x \\
& H_0 : \mu = x, H_A: \mu \neq x
\end{align*}

\begin{enumerate}
\item meistens $p=0,5$, sonst $\hat{p}$ berechnen
\item $SE$ berechnen
\item $Z=\frac{\hat{p}-p}{SE}$
\item Falls Wahrscheinlichkeit gefragt: $2 \cdot (1-P(X \leq |Z|)$
\end{enumerate}

\subsection{t-Verteilung}
Inferenz für numerische Variablen, bei Stichproben $n < 30$

\begin{enumerate}
\item Unterschiede berechnen, wenn nicht gegeben: $\bar{x} = \bar{x}_A - \bar{x}_b$
\item Standardfehler: $SE = \frac{s}{\sqrt{n}}$
\item Standardfehler bei Gruppenunterscheidung: $SE = \sqrt{\frac{s_A^2}{n_A} + \frac{s_B^2}{n_B}}$
\item Teststatistik: $T = \frac{\bar{x} - \text{null value}}{SE}$
\item Freiheitsgrad (immer kleineres n, z.B. $n_A < n_B$): $df = n - 1$
\item p-Wert gefragt: $p=2 \cdot (1-P(X \leq |T|)$
\item Intervall gefragt: $estimate \pm x \cdot SE$
\end{enumerate}

\subsection{Chi-Quadrat-Test}
\[\resizebox{0.25\textwidth}{!}{$
\chi^2 = \sum \frac{(\text{observed count}_i-\text{null count}_i)^2}{\text{null count}_i} = \sum Z_i, ~\text{mit } Z_i = \frac{\text{observed} - \text{expected counts}}{\sqrt{\text{expected counts}}}
$}
\]

\subsection{Inferenz}
\begin{enumerate}
\item $H_0: \beta_1=0$ zugunsten $H_A: \beta_1 \neq 0$ ablehnen
\item $\hat{y}= \text{Intercept} + \text{Steigung}x$
\item $T = \frac{\text{estimate} - \text{null value}}{\text{SE}}$
\end{enumerate}

%=========================================================================================================================================================================================================================

\section{Beispielaufgaben}

\subsection{Bayes-Aufgabe}
Gegeben: $P(L) = 0,02$, $P(Pos|L)$, $P(\bar{Pos}|\bar{L})$

\begin{align*}
& P (\bar{L}) 1 - P(L) \\
& P (Pos|\bar{L}) = 1 - P(\bar{Pos}|\bar{L}) \\
& P (L|Pos) = \frac{P(Pos|L)P(L)}{P(Pos|L)P(L)+P(Pos|\bar{L})P(\bar{L})}
\end{align*}

\subsection{P(X>Z) zu Z}
\begin{align*}
& P(X > Z) = x\\
& \Phi(Z) = P(X \leq Z) = 1- P(X > Z) = x
\end{align*}

\subsection{$\hat{p}$-Wert ohne $\frac{x}{n}$}
Gegeben: $p = a$ und wichtig $\hat{p} = Z \cdot SE + p$

\begin{align*}
& a = 2 \cdot (1 - P(X < |Z|) \\
\Leftrightarrow & \frac{a}{2} = 1 - P(X < |Z|) \\
\Leftrightarrow & P(X < |Z|) = 1 - \frac{a}{2} \\
\Rightarrow & Z = b \Rightarrow \hat{p} = Z \cdot SE +p
\end{align*}

\subsection{$\chi^2$ bei Hypothesentest}
Gegeben: meistens Kontigenztafel mit absoluten Häufigkeiten

\begin{enumerate}
\item Hypothesen aufstellen
\item Erwartungswerte berechnen: $E_{xy}=\frac{h_{x \cdot} \cdot h_{\cdot y}}{n}$
\item $\chi^2$ berechnen
\item Freiheitsgrad $df$ berechnen
\item $p = 1 - F(\chi^2_{df})$
\end{enumerate}

%=========================================================================================================================================================================================================================

\section{Entscheidungshilfe}
\includegraphics[scale=0.15]{Bilder/1.png} 
\includegraphics[scale=0.15]{Bilder/2.png} 

\end{multicols}
\end{document} 